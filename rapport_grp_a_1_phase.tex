\documentclass[a4paper,12pt]{article}
\usepackage[utf8]{inputenc}
\usepackage[T1]{fontenc}
\usepackage[frenchb]{babel}

\begin{document}

\begin{titlepage}
\begin{center}

{\large ENSEEIHT}\\[0.5cm]

{\large 2A IMA}\\[0.5cm]

\rule{\linewidth}{0.5mm} \\[0.4cm]
{ \huge \bfseries Systèmes concurrents, première partie, groupe A \\[0.4cm] }
\rule{\linewidth}{0.5mm} \\[1.5cm]

\noindent
\begin{minipage}{0.4\textwidth}
  \begin{flushleft} \large
    \emph{Auteurs :}\\
    Lucien \textsc{Haurat}\\
    Noah \textsc{Jay}
  \end{flushleft}
\end{minipage}%

\vfill

{\large \today}

\end{center}
\end{titlepage}


\clearpage
\tableofcontents
\newpage

\section{Principe}
La classe \texttt{CentralizedLinda} est assez simple, elle contient deux listes, et deux \texttt{ReentrantLock}, une pour chaque liste. A chaque fois que le programme a besoin d'utiliser, en lecture ou en écriture, une des listes, la fonction récupère le verrou avant.

La première liste est la liste des tuples stockés.
La seconde liste est une liste de \texttt{CallbackRef}, qui stocke un \texttt{Callback}, son timing, et son action.

\section{Plus de détails}
Pour une fonction bloquante de lecture (\texttt{take} ou \texttt{read}), le thread est mis en attente grâce a l'appel de la fonction \texttt{wait}. Lors d'un appel à la fonction \texttt{write}, on notifie tous les threads en attente de lecture, pour qu'ils terminent, si le tuple ajouté correspond à leur template.
A chaque appel à \texttt{read} ou à \texttt{write}, on crée une copie profonde du tuple, afin de ne pas stocker l'objet passé en paramètre, mais une copie de celui-ci.

De plus, lors d'un appel à \texttt{write}, on appelle la fonciton \texttt{checkCallbacks}, qui parcourt toutes les opérations en attente, et leur transmet le tuple écrit s'il correspond à ce qu'ils attendent.
Ceci permet de prendre en compte les callbacks de type \texttt{FUTURE}, et ceux du type \texttt{IMMEDIATE} qui sont en attente.
Les callbacks de type \texttt{IMMEDIATE} sont eux traités quand on enregistre un callback avec la méthode \texttt{eventRegister}, et sont mis en attente si on ne trouve pas de tuple qui correspond au template fourni.

\end{document}

